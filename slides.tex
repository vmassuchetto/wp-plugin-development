\documentclass{beamer}

\usepackage[brazilian]{babel}
\usepackage[utf8]{inputenc}
\usepackage{graphicx}
\usepackage{fontenc}
\usepackage{listings}
\usepackage{verbatim}
\usepackage{pxfonts}
\usepackage{graphicx}
\usetheme{Warsaw}

\title{Plugins no WordPress}
\subtitle{Fazendo o Negócio Direito}
\author{Vinicius Massuchetto}
\institute{WordCamp São Paulo 2012}
\date{Agosto de 2012}

\lstset{%
  breakatwhitespace,
  columns=fullflexible,
  keepspaces,
  breaklines,
  tabsize=2,
  showstringspaces=false,
  extendedchars=true,
  basicstyle=\footnotesize\ttfamily,
  frame=leftline}

\begin{document}

\frame{\titlepage}

\section{Introdução}

\subsection{Download}

\begin{frame}{Download}
  \begin{center}
    Esta apresentação está disponível em: \\
    \url{http://vinicius.soylocoporti.org.br/?p=2191}
  \end{center}
\end{frame}

\subsection{Sobre a Palestra}

\begin{frame}{Sobre a Palestra}
\begin{itemize}
  \item Motivação Desta Apresentação
\end{itemize}
\end{frame}

\subsection{Motivação}

\begin{frame}{Não encontrar o que procura}
  \begin{center}
    \includegraphics[height=0.8\textheight]{./img/plugins.jpg}
  \end{center}
\end{frame}

\begin{frame}{Motivos para se criar um plugin}
\begin{itemize}
  \pause \item Funcionalidade inexistente
  \pause \item Diferente implementação de uma funcionalidade existente
  \pause \item Códigos de tema portáveis
  \pause \item Implementações modulares para clientes
  \pause \item Forks de plugins existentes
\end{itemize}
\end{frame}

\begin{frame}{Dificuldades em se escrever um plugin}
\begin{itemize}
  \pause \item PHP X WordPress
  \pause \item Cultura de leitura de documentação e inspeção de código
  \pause \item Barreira de idioma
  \pause \item Entrega de produtos deficitários
\end{itemize}
\end{frame}

\begin{frame}{Facilidades}
\begin{itemize}
  \pause \item Código legível
  \pause \item Padronização de estruturas
  \pause \item Melhor aprendizado e manutenção do código
  \pause \item Extensibilidade
  \pause \item Distributividade na comunidade do software livre
\end{itemize}
\end{frame}

\subsection{Avançando a Ideia}

\begin{frame}{Pensando Em Um Plugin}
\begin{itemize}
  \pause \item Definição de escopo e pesquisa de funcionalidades
  \pause \item Se parecer redundante, perguntar e descrever a
    ideia em listas e fóruns
  \pause \item Escolha de nome único e relevante
  \pause \item Avaliação do uso de outras tecnologias e frameworks
\end{itemize}
\end{frame}

\section{Padrões}

\begin{frame}\end{frame}

\subsection{Primeiro Padrão}

\begin{frame}{Primeiro Padrão}
  \begin{center}
    \includegraphics[height=0.8\textheight]{./img/hack-core.jpg}
  \end{center}
\end{frame}

\subsection{Arquivos}

\begin{frame}{Arquivos}
\begin{itemize}
  \pause \item Nomear o diretório
  \pause \item Ser coerente com nomes de arquivos
  \pause \item Incluir somente arquivos necessários e sob demanda no código
  \pause \item Permitir que o diretório do plugin mude usando funções como:
  \begin{itemize}
    \pause \item \texttt{plugins\_url}
    \pause \item \texttt{plugin\_dir\_url}
    \pause \item \texttt{plugin\_dir\_path}
  \end{itemize}
\end{itemize}
\end{frame}

\subsection{Padrões de Código}

\begin{frame}{Padrões de Código}
\begin{itemize}
  \pause \item Ater-se aos padrões recomendados para código e documentação
  \pause \item Nomear as estruturas e funções com um identificador único
  \pause \item Clareza é melhor do que praticidade
\end{itemize}
\end{frame}

\begin{frame}{Tag PHP}
  \pause Errado
  \lstinputlisting{./code/standards-php-tag-wrong.php}
\end{frame}

\begin{frame}{Tag PHP}
  Certo
  \lstinputlisting{./code/standards-php-tag-right.php}
\end{frame}

\begin{frame}{Chaves}
  Errado
  \lstinputlisting{./code/standards-brace-wrong.php}
\end{frame}

\begin{frame}{Chaves}
  Certo
  \lstinputlisting{./code/standards-brace-right.php}
\end{frame}

\begin{frame}{Funções}
  Errado
  \lstinputlisting{./code/standards-spaces-wrong.php}
\end{frame}

\begin{frame}{Funções}
  Certo
  \lstinputlisting{./code/standards-spaces-right.php}
\end{frame}

\begin{frame}{Vetores}
  \lstinputlisting{./code/standards-arrays.php}
\end{frame}

\subsection{Padrões de SQL}

\begin{frame}{Padrões de SQL}
\begin{itemize}
  \pause \item Evitar escrever consultas
  \pause \item Validar os tipos de dados antes de utilizá-los
  \pause \item Selecionar só o que for necessário
  \pause \item Utilizar a \texttt{wpdb}
  \pause \item Se precisar criar tabelas no banco, use \texttt{\$wpdb->prefix}
\end{itemize}
\end{frame}

\begin{frame}{Exemplo de Consulta}
  Errado
  \lstinputlisting{./code/standards-sql-wrong.php}
\end{frame}

\begin{frame}{Exemplo de Consulta}
  Certo
  \lstinputlisting{./code/standards-sql-right.php}
\end{frame}

\section{Desenvolvimento}

\begin{frame}\end{frame}

\begin{frame}{Debug}
  \begin{center}
    \includegraphics[height=0.8\textheight]{./img/debug.jpg}
  \end{center}
\end{frame}

\begin{frame}{Constantes de debug no \texttt{wp-config.php}}
\begin{itemize}
  \item \pause \texttt{WP\_DEBUG}
  \item \pause \texttt{WP\_DEBUG\_LOG}
  \item \pause \texttt{WP\_DEBUG\_DISPLAY}
\end{itemize}
\end{frame}

\begin{frame}{Cabeçalho}
  \pause Todo plugin começa pelo começo..
  \pause \lstinputlisting{./code/plugin-header.txt}
\end{frame}

\subsection{Estrutura}

\begin{frame}{Estrutura Procedural}
  \pause \lstinputlisting{./code/estrutura-procedural.php}
\end{frame}

\begin{frame}{Estrutura Orientada a Objetos}
  \pause \lstinputlisting{./code/estrutura-oo.php}
\end{frame}

\begin{frame}{Vantagens da Orientação a Objetos em Plugins}
\begin{itemize}
  \pause \item Organiza o código, suas funções e métodos
  \pause \item Melhora a extensibilidade
  \pause \item Reduz o impacto no escopo global do PHP
  \pause \item Ajuda a não introduzir variáveis globais
\end{itemize}
\end{frame}

\subsection{Interfaces}

\begin{frame}{Ativação}
\begin{itemize}
  \pause \item \texttt{register\_activation\_hook}
  \pause \item Criação de opções padrão
  \pause \item Criação de tabelas
  \pause \item Exibição de avisos para o usuário configurar o plugin
\end{itemize}
\end{frame}

\begin{frame}{Desativação}
\begin{itemize}
  \pause \item \texttt{register\_deactivation\_hook}
  \pause \item Em geral não deve causar nenhuma perda de dados
  \pause \item Desabilitar plugins dependentes
\end{itemize}
\end{frame}

\begin{frame}{Desinstalação}
\begin{itemize}
  \pause \item \texttt{register\_uninstall\_hook}
  \pause \item Não deve deixar nenhum dado residual no WordPress
  \pause \item Remove opções do usuário
  \pause \item Remove tabelas
  \pause \item Avisa o usuário sobre a remoção de dados
\end{itemize}
\end{frame}

\begin{frame}{Inicialização}
\begin{itemize}
  \pause \item \texttt{*\_init()}
  \pause \item Geralmente através de um procedimento inicializador
\end{itemize}
\end{frame}

\begin{frame}{Inicialização}
  \lstinputlisting{./code/plugin-init-procedure-inline.php}
  \pause \lstinputlisting{./code/plugin-init-procedure-pl.php}
\end{frame}

\begin{frame}{Inicialização}
  \lstinputlisting{./code/plugin-init-oo-inline.php}
  \pause \lstinputlisting{./code/plugin-init-oo-pl.php}
\end{frame}

\begin{frame}{Banco de dados}
Usar sempre a \texttt{wpdb}:
\begin{itemize}
  \pause \item \texttt{query}
  \pause \item \texttt{prepare}
  \pause \item \texttt{insert}
  \pause \item \texttt{update}
  \pause \item \texttt{get\_var}
\end{itemize}
\end{frame}

\begin{frame}{Tratando dados para consultas}
  \pause \lstinputlisting{./code/standards-sql-prepare.php}
\end{frame}

\begin{frame}{Uso de Ações e Filtros}
\begin{itemize}
  \pause \item Base da construção de plugins no WordPress
  \pause \item Certificar-se de agendar os eventos e
    tratar os dados adequadamente
\end{itemize}
\end{frame}

\begin{frame}{Implementação de Ações e Filtros}
\begin{itemize}
  \pause \item Oferecer extensibilidade aos dados gerados
  \pause \item Possibilitar a inserção de novos procedimentos à medida
    que eventos relevantes acontecem
\end{itemize}
\end{frame}

\begin{frame}{Implementação de Ações}
  \lstinputlisting{./code/plugin-ext-action.php}
  \pause \lstinputlisting{./code/plugin-ext-action-use.php}
\end{frame}

\begin{frame}{Implementação de Filtros}
  \lstinputlisting{./code/plugin-ext-filter.php}
  \pause \lstinputlisting{./code/plugin-ext-filter-use.php}
\end{frame}

\begin{frame}{Implementação de Ações e Filtros}
  \lstinputlisting{./code/plugin-ext-form.php}
\end{frame}

\subsection{Scripts e Estilos}

\begin{frame}\end{frame}

\begin{frame}{Enfileiradores de scripts}
\pause Funções:
\begin{itemize}
  \pause \item \texttt{wp\_enqueue\_script}
  \pause \item \texttt{wp\_enqueue\_style}
  \pause \item \texttt{wp\_localize\_script}
\end{itemize}
\pause Hooks:
\begin{itemize}
  \pause \item \texttt{wp\_enqueue\_scripts}
  \pause \item \texttt{admin\_enqueue\_scripts}
\end{itemize}
\end{frame}

\begin{frame}{Incluindo scripts}
  \pause \lstinputlisting{./code/scripts-enqueue.php}
\end{frame}

\begin{frame}{Incluindo scripts e variáveis}
  \pause \lstinputlisting{./code/scripts-enqueue-localized.php}
\end{frame}

\begin{frame}{Incluindo scripts e variáveis}
  \pause \lstinputlisting{./code/scripts-enqueue-localized-result.txt}
\end{frame}

\subsection{Ferramentas}

\begin{frame}\end{frame}

\begin{frame}{Não precisa reinventar a roda}
  \begin{center}
    \includegraphics[height=0.8\textheight]{./img/rewrite.jpg}
  \end{center}
\end{frame}

\begin{frame}{Internacionalização}
\begin{itemize}
  \pause \item Usar funções \texttt{\_\_()} e \texttt{\_e()}
  \pause \item Carregar o arquivo MO de idioma
\end{itemize}
\end{frame}

\begin{frame}{Tratamento de erros}
\begin{itemize}
  \pause \item \texttt{WP\_Error}
\end{itemize}
\end{frame}

\begin{frame}{Morrer elegantemente}
\begin{itemize}
  \pause \item \texttt{wp\_die}
\end{itemize}
\end{frame}

\begin{frame}{Checar compatibilidade com versão}
  \pause if (!function\_exists) ..
\end{frame}

\begin{frame}{Formatação}
\begin{itemize}
  \pause \item \texttt{is\_email}
  \pause \item \texttt{remove\_accents}
  \pause \item \texttt{sanitize\_title}
  \pause \item \texttt{sanitize\_email}
  \pause \item \texttt{seems\_utf8}
  \pause \item \texttt{zeroise}
  \pause \item \texttt{wptexturize}
\end{itemize}
\end{frame}

\begin{frame}{Transients API}
\begin{itemize}
  \pause \item \texttt{set\_transient}
  \pause \item \texttt{get\_transient}
  \pause \item \texttt{delete\_transient}
\end{itemize}
\end{frame}

\begin{frame}{HTTP API}
\begin{itemize}
  \pause \item \texttt{wp\_remote\_get}
  \pause \item \texttt{wp\_remote\_retrieve\_body}
  \pause \item \texttt{wp\_remote\_retrieve\_headers}
\end{itemize}
\end{frame}

\begin{frame}{Object Cache}
\begin{itemize}
  \pause \item \texttt{wp\_cache\_add}
  \pause \item \texttt{wp\_cache\_set}
  \pause \item \texttt{wp\_cache\_get}
  \pause \item \texttt{wp\_cache\_delete}
  \pause \item \texttt{wp\_cache\_flush}
\end{itemize}
\end{frame}

\begin{frame}{Classes Úteis}
\begin{itemize}
  \pause \item \texttt{SimplePie}
  \pause \item \texttt{PHPMailer}
\end{itemize}
\end{frame}

\begin{frame}{Funções Úteis}
\begin{itemize}
  \pause \item \texttt{wp\_mail}
  \pause \item \texttt{fetch\_feed}
  \pause \item \texttt{human\_time\_diff}
\end{itemize}
\end{frame}


































\section{Considerações Finais}

\begin{frame}{Referências}
\begin{itemize}
  \item Codex: Writing a Plugins \\
    \url{http://codex.wordpress.org/Writing_a_Plugin}
  \item WordPress Answers
    \url{http://wordpress.stackexchange.com/questions/715/objective-best-practices-for-plugin-development}
\end{itemize}
\end{frame}

\end{document}
